\documentclass{bmcart}

\usepackage[utf8]{inputenc} 

\def\includegraphic{}
\def\includegraphics{}
\startlocaldefs
\endlocaldefs

\begin{document}
	
	\begin{frontmatter}
		
		\dochead{Documento de Investigacion}
		
		\title{El Tabaquismo y sus Consecuencias}
		
		\author[
		]{\inits{HG}\fnm{Hernandez Gilbert} \snm{Manuel Andres}}
		
	\end{frontmatter}
	
	\section*{Motivos para llevar a cabo esta investigacion}
	En esta investigacion se lleva a cabo debido a que, aunque la mayoria de las personas estan al tanto de que el tabaco puede ser causante de cancer, no estan informadas de las muchas otras consecuencias que puede traer a sus organismos el consumo a largo plazo de este producto. Con esto entonces, el objetivo principal es dar a conocer los demas peligros que puede traer el tabaco y con la esperanza de evitar que mas personas sufran de las consecuencias que ocasiona, ademas de explorar las causas que puede haber en la vida de las personas para comenzar a fummar en su vida diaria. 
	
	\section*{Introduccion}
	El tabaquismo es la dependencia o adicción al tabaco. Es la adicción más aceptada socialmente, pero que representa una de las principales causas prevenibles de muerte en el mundo.
	Durante las últimas décadas, el tabaquismo se ha convertido en uno de los problemas de salud más severos, al descubrirse los daños que ocasiona y al encontrarlo relacionado con diversas enfermedades graves, que cada día adquieren mayor importancia como causa de muerte e incapacidad, con severas repercusiones económicas.
	Fumar, además de los riesgos a la salud, que son muchos, está asociado a un sinnúmero de enfermedades crónicas. Se asocia también a otro tipo de problemas como: incendios, quemaduras y accidentes de tráfico.
	Esta enfermedad, considerada como una adicción de riesgo voluntario, es muy difícil de abandonar y controlar, por lo que una vez iniciado el hábito es sumamente difícil de dejarlo, ya que pasa a ser parte del estilo de vida de una persona, quien a pesar de saber el daño que se hace, no se da cuenta que a cambio de un “bienestar” pasajero, en forma lenta, silenciosa, pero efectiva, el tabaco va ocasionando daños irreversibles en la mayoría de los órganos del cuerpo, generando varias enfermedades crónicas y degenerativas y es causa de muerte prematura.
	
	\section*{Que nos puede provocar}
	El hábito de fumar perjudica a casi todos los órganos del cuerpo. Ha sido definitivamente vinculado a las cataratas y la neumonía (pulmonía) y ocasiona la tercera parte de las muertes relacionadas con cualquier tipo de cáncer. Además, el uso de cigarrillos causa enfermedades pulmonares como la bronquitis crónica y el enfisema y también se sabe que empeora los síntomas del asma en adultos y niños. Más del 90 por ciento de todas las muertes por enfermedades de obstrucción pulmonar crónica son atribuibles al hábito de fumar. Ha sido bien documentado que fumar aumenta sustancialmente el riesgo de enfermedades del corazón, incluyendo ataques al cerebro y al corazón, risma. El hábito de fumar cigarrillos también causa enfermedad coronaria, la principal causa de muerte en los Estados Unidos; los fumadores de cigarrillos tienen de dos a cuatro veces más probabilidad de desarrollar enfermedad coronaria en comparación con las personas que no fuman. Aunque los efectos del tabaco se producen de forma progresiva y están directamente relacionados con el tiempo de duración de la adicción, las consecuencias son demoledoras desde la primera calada: la nicotina alcanza el cerebro poco tiempo después de ser consumida, el monóxido de carbono impide que los glóbulos rojos puedan realizar su función de transporte de oxígeno a todos los órganos del cuerpo. ¿Por qué ocurre esto?. Porque la sangre es la encargada (entre otras cosas) de transportar por todo el cuerpo el oxígeno que recoje en los pulmones. Al fumar, los pulmones se llenan de humo con cientos de sustancias tóxicas, que al no haber oxígeno, son transportadas y expandidas por el sistema circulatorio. Por ello, todos los órganos del cuerpo se ven deteriorados en un fumador, porque reciben mucho menos oxígeno que el resto de las personas.
	Además, los productos cancerígenos que contiene el tabaco dañan el A.D.N. de las células, alterando su microambiente y desencadenando la aparición y expansión de múltiples tumores.
	
	\section*{¿Por qué las personas fuman?}
	las razones para fumar son principalmente psicológicas (pero no en su totalidad) aspectos que influyen mucho en estos casos es la publicidad, muchos individuos son “seducidos” por el uso del cigarrillo en películas o anuncios comerciales, una vez que el hábito es creado, la nicotina crea una cierta dependencia física, sin embargo, ten en cuenta que nos referimos a aquellos cigarrillos comerciales que poseen un alto contenido de químicos nocivos que generan un efecto más poderoso de dependencia (alquitrán, monóxido de carbono, gas cianhidríco, etc) y además, el nivel de nicotina de estos es más elevado.
	
	El consumo del cigarrillo podría llegar a convertirse en hábito desde la adolescencia, de hecho, según un estudio realizado por la American Cancer Society, son pocas las personas que comienzan a fumar después de los 25, 9 de cada 10 fumadores comenzaron a partir de los 18 años. Las industrias tabacaleras gastan millones de dólares cada año para crear publicidades llamativas y muy bien hechas, sin embargo, estas lo que buscan es enriquecerse, vendiendo productos que poseen químicos altamente dañinos y no son naturales, a diferencia de las hojas de tabaco, que han servido desde milenios como un hábito cultural de distintas civilizaciones y son más saludables.
	
	Son varias las razones por las cuales las personas empiezan a fumar:
	
	Los jóvenes lo usan para demostrar que no son tímidos o no tienen miedo.
	Para adaptarse a un círculo social.
	Para afirmar independencia.
	Por la influencia de personas cercanas (amigos, padres, famosos, deportistas, etc.)
	Por influencia de la publicidad.
	Para calmar los nervios.
	Para bajar de peso.
	Algunas razones que perpetuán el hábito
	Ya conoces las razones por la cual las personas empiezan a fumar, pero una vez que agarrán el “gusto” por el cigarrillo, continúan este hábito por distintos motivos, uno de los cuales es el ámbito social, que ejerce una presión e influye en los individuos, sin embargo, los siguientes puntos responderán a tu pregunta de ¿por qué las personas fuman cigarrillos?:
	
	Muchos individuos fuman cigarrillos para relajarse, como una vía de escape o para tener un momento para ellos mismos, así alejan las preocupaciones.
	Para sociabilizar, ya que fumar con otras personas es visto como una “actividad compartida” y con el cigarrillo se “dan valor” para iniciar alguna conversación, los hace sentir más confiados y con valor.
	La adicción física tiene mucho que ver y la nicotina cumple un rol protagonista, sin embargo, no todos los que fuman sufren de una dependencia causada por este químico, los fumadores de hojas de tabaco no suelen tener estos problemas.
	Muchos fuman para mejorar sus ánimos, el cigarrillo los ayuda a continuar su rutina o les da ánimos cuando se aburren.
	
	\section*{Referencias}
	$
	http://www.contabaco.com/blog/21_por-que-las-personas-fuman.html\\
	http://www.eltabacoapesta.com/el-tabaco/efectos-del-tabaco/consecuencias-del-tabaco/\\
	https://www.drugabuse.gov/es/publicaciones/serie-de-reportes/adiccion-al-tabaco/cuales-son-las-consecuencias-medicas-del-uso-del-tabaco\\
	http://www.monografias.com/trabajos11/tabac/tabac.shtml\\
	https://www.drugabuse.gov/es/publicaciones/serie-de-reportes/adiccion-al-tabaco/el-tabaquismo-y-los-adolescentes\\
	http://www.dalinde.com/tabaquismo/educacion/tabaquismo.html\\
	http://www.monografias.com/trabajos66/tabaquismo/tabaquismo.shtml\\
	$
\end{document}